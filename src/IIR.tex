\chapter{Disposizione di poli e zeri nei filtri IIR}
I filtri IIR (Infinitive Impulse Response) sono filtri appartenenti alla sottoclasse dei sistemi LTI causali caratterizzati da una risposta infinita all'impulso.\\
Questi filtri vengono rappresentati nel dominio delle frequenze con una funzione di trasferimento razionale in $z^{-1}$ della forma
\begin{displaymath}
H(z)=\frac{\sum_{k=0}^{M-1} b_kz^{-k}}{\sum_{k=0}^{L-1} a_kz^{-k}}
\end{displaymath}
Utilizzando Scilab, \`e possibile progettare un filtro IIR ed ottenerne i coefficienti grazie alla funzione \verb+iir+, dove
\begin{itemize}
\item l'ordine del filtro
\item il tipo di filtro: \verb+lp+ (passa-basso), \verb+hp+ (passa-alto), \verb+bp+ (passa-banda), \verb+sb+ (taglia-banda)
\item il corrispondente design analogico: \verb+butt+ (Butterworth), \verb+cheb1+ (Chebyschev, oscillazioni solo in banda passante), \verb+cheb2+ (Chebyschev, oscillazioni solo in banda proibita), \verb+ellip+ (ellittico)
\item una o due frequenze di taglio, in base al tipo di filtro scelto
\item gli errori ammessi nelle bande definite
\end{itemize}

\section{Filtri di Butterworth}
I filtri di Butterworth sono caratterizzati da un discreto guadagno in banda passante, da un sufficiente guadagno in banda di transizione e da una fase non lineare ma comunque di discreta qualit\`a.\\
La forma generale del modulo del filtro di Butterworth \`e del tipo
\begin{displaymath}
\frac{1}{\left|B_N \left(\frac{\omega}{\omega_c}\right)\right|}=\frac{1}{\sqrt{1+\left(\frac{\omega}{\omega_c}\right)^{2N}}}
\end{displaymath}
dove $N$ \`e l'ordine del filtro, $\omega_c$ \`e la frequenza di taglio e $B_N$ \`e il polinomio $N-esimo$ di Butterworth.\\
Creiamo un filtro Butterworth passa-basso di ordine 3 con frequenza di taglio 0,15 e frequenza di stop 0,3.\\
Il diagramma dei poli/zeri e della risposta in frequenza pu\`o essere ottenuto in Scilab con i seguenti comandi
\begin{verbatim}
// filtro BUTTERWORTH
[hz]=iir(3,'lp','butt',[0.125 .3],[0.2 0.2]);
[hzm,fr]=frmag(hz,100);
xbasc();xset('window',0);
// grafico risposta in frequenza
plot2d(fr,hzm,rect=[0,0,.5,1.1]);
xtitle(tit="",xax="frequenza",yax="|H(w)|")
// grafico poli/zeri
xset('window', 1);
plzr(hz);
\end{verbatim}
Otteniamo cos\`i il risultato di figura \ref{Butterworth-fr} e figura \ref{Butterworth-pz}.

\input{images/Butterworth-pz.tex}
\dessin{Grafico poli/zeri del filtro di Butterworth}{Butterworth-pz}
\input{images/Butterworth-fr.tex}
\dessin{Risposta in frequenza del filtro di Butterworth}{Butterworth-fr}

\section{Filtri di Chebyschev}
La caratteristica principale dei filtri di Chebyschev \`e quella di distribuire l'accuratezza dell'approssimazione uniformemente lungo tutta la banda passante (tipo I) o quella proibita (tipo II) scegliendo un'approssimazione con oscillazioni della stessa ampiezza su tutta la banda.\\
La forma generale del modulo del filtro di Chebyschev \`e del tipo
\begin{displaymath}
\frac{1}{\sqrt{1+\epsilon^2C^{2}_{N}\left(\frac{\omega}{\omega_c}\right)}}
\end{displaymath}
dove $N$ \`e l'ordine del filtro, $\omega_c$ \`e la frequenza di taglio, $\epsilon$ \`e una costante minore di 1 e $C_N$ \`e il polinomio $N-esimo$ di Chebyschev.
\subsection{Filtro di Chebyschev - tipo I}
Creiamo un filtro di Chebyschev taglia-banda di ordine 10 con frequenza di taglio di 0,2 e 0,35.\\
Il diagramma dei poli/zeri e della risposta in frequenza pu\`o essere ottenuto in Scilab con i seguenti comandi
\begin{verbatim}
// filtro CHEBYTCHEV tipo I
[hz]=iir(10,'sb','cheb1',[.2 .35],[0.1 0.1]);
[hzm,fr]=frmag(hz,100);
xbasc();xset('window',0);
// grafico risposta in frequenza
plot2d(fr,hzm,rect=[0,0,.5,1.1]);
xtitle(tit="",xax="frequenza",yax="|H(w)|")
// grafico poli/zeri
xset('window', 1);
plzr(hz);
\end{verbatim}
Otteniamo cos\`i il risultato di figura \ref{Chebyschev_1-fr} e figura \ref{Chebyschev_1-pz}.
\input{images/Chebyschev_1-pz.tex}
\dessin{Grafico poli/zeri del filtro di Chebyschev tipo I}{Chebyschev_1-pz}
\input{images/Chebyschev_1-fr.tex}
\dessin{Risposta in frequenza del filtro di Chebyschev tipo I}{Chebyschev_1-fr}

\subsection{Filtro di Chebyschev - tipo II}
Creiamo un filtro di Chebyschev passa-basso di ordine 8 con frequenza di taglio 0,15 e frequenza di stop 0,25.\\
Il diagramma dei poli/zeri e della risposta in frequenza pu\`o essere ottenuto in Scilab con i seguenti comandi
\begin{verbatim}
// filtro CHSBYSCHEV tipo II
[hz]=iir(8,'lp','cheb2',[.15 .25],[0.1 0.1]);
[hzm,fr]=frmag(hz,1000);
xbasc();xset('window',0);
// grafico risposta in frequenza
plot2d(fr,hzm,rect=[0,0,.5,1.05]);
xtitle(tit="",xax="frequenza",yax="|H(w)|");
// grafico poli/zeri
xset('window', 1);
plzr(hz);
\end{verbatim}
Otteniamo cos\`i il risultato di figura \ref{Chebyschev_2-fr} e figura \ref{Chebyschev_2-pz}.
\input{images/Chebyschev_2-pz.tex}
\dessin{Grafico poli/zeri del filtro Chebyschev tipo II}{Chebyschev_2-pz}
\input{images/Chebyschev_2-fr.tex}
\dessin{Risposta in frequenza del filtro Chebyschev tipo II}{Chebyschev_2-fr}

\section{Filtri ellittici}
I filtri ellittici sono caratterizzati da un ottimo guadagno sia in banda passante che in banda di transizione ma da una fase non lineare di pessima qualit\`a.\\
La forma generale del modulo del filtro ellittico \`e del tipo
\begin{displaymath}
\frac{1}{\sqrt{1+\epsilon^2{R_n}^2(\omega)}}
\end{displaymath}
Creiamo un filtro ellittico passa-basso di ordine 4 con frequenza di taglio 0,15 e frequenza di stop 0,25.\\
Il diagramma dei poli/zeri e della risposta in frequenza pu\`o essere ottenuto in Scilab con i seguenti comandi:
\begin{verbatim}
// filtro ELLITTICO
[hz]=iir(4,'lp','ellip',[.15 .25],[0.1 0.1]);
[hzm,fr]=frmag(hz,1000);
xbasc();xset('window',0);
// grafico risposta in freqenza
plot2d(fr,hzm,rect=[0,0,.5,1.1]);
xtitle(tit="",xax="frequenza",yax="|H(w)|");
// grafico poli/zeri
xset('window', 1);
plzr(hz);
\end{verbatim}
Otteniamo cos\`i il risultato di figura \ref{Ellittico-fr} e figura \ref{Ellittico-pz}.\\
\input{images/Ellittico-pz.tex}
\dessin{Grafico poli/zeri del filtro Ellittico}{Ellittico-pz}
\input{images/Ellittico-fr.tex} \dessin{Risposta in frequenza del filtro Ellittico}{Ellittico-fr}
%\end{document}
