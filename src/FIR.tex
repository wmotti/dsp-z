%%%%% Preamble %%%%%
%\documentclass[a4paper,11pt]{article}
%\author{Walter Mottinelli}
%\title{Elaborazione numerica dei segnali con Scilab}
%\usepackage{latexsym}
%\usepackage{makeidx} % to generate indexes
%\usepackage[italian]{babel}
%\usepackage{ucs} % unicode encoding
%\usepackage[utf8]{inputenc}
%\usepackage{graphicx} % eps importing
%\pagestyle{headings}
%\newcommand{\Figdir}{images/}
%%%%%%%%%%%%%%%%%%%%
                                                                           
%\begin{document}
%\newcommand{\Figdir}{images/}
%\maketitle
%\tableofcontents
%\begin{abstract}
%\end{abstract}

\chapter{Disposizione di poli e zeri nei filtri FIR}
I filtri FIR sono listemi LTI aventi risposta finita all'impulso e sono rappresentati da un polinomio in $z^{-1}$:
\begin{displaymath}
H(z)=\sum_{k=0}^{M-1}h(k)z^{-k}
\end{displaymath}
Il diagramma poli/zeri corrispondente presenta un polo in $z=0$, quindi interno alla circonferenza di raggio unitario: i filtri FIR sono sempre causali e stabili, inoltre hanno fase lineare se la funzione di risposta all'impulso \`e simmetrica o antisimmetrica.\\
Vediamone qualche esempio, usando Scilab come strumento di analisi.

\section{Filtro con uno zero sulla circonferenza unitaria}
Verifichiamo innanzitutto che uno zero in $z=e^{i\omega_0}$ modifica la risposta in frequenza azzerandola nella frequenza corrispondente a $\omega_0$.
Un filtro di questo tipo ha la funzione di trasferimento $H(z)=\frac{1-z^{-1}}{z^{-1}}$ e il diagramma poli/zeri pu\`o essere ottenuto in Scilab con i seguenti comandi:
\begin{verbatim}
// inizializziamo il sistema
clear;
// creiamo la funzione di trasferimento:
// definiamo il vettore di coefficienti del numeratore
numcoeff=[1 -1];
// creiamo il polinomio al numeratore
num=poly(numcoeff,'z','c');
// definiamo il vettore di coefficienti del denominatore
dencoeff=[0 1];
// creiamo il polinomio al denominatore
den=poly(dencoeff,'z','c');
// creiamo la funzione di trasferimento
combsl=syslin('d',num,den);
// inizializziamo la finestra grafica
xbasc();
// visualizziamo il grafico poli/zeri
plzr(combsl);
\end{verbatim}
Otteniamo il risultato di figura \ref{fir-1zero-pz}.
\input{images/fir-1zero-pz.tex}
\dessin{Grafico poli/zeri}{fir-1zero-pz}

Calcoliamo ora la risposta in frequenza:
\begin{verbatim}
// definiamo il range discreto delle frequenze
f=(-.5:.001:.5);
// calcoliamo la risposta in frequenza
hf=freq(num,den,exp(2*%pi*%i*f));
// visualizziamo il grafico
xbasc(); plot2d(f,abs(hf))
\end{verbatim}
Il filtro ha la risposta in frequenza attesa, come dimostra il grafico \ref{fir-1zero-fr}.
\input{images/fir-1zero-fr.tex}
\dessin{Risposta in frequenza}{fir-1zero-fr}

\section{Filtri COMB}
Un filtro COMB di ordine M ha funzione di trasferimento 
\begin{displaymath}
H(z)=\frac{1}{M}(1-z^{-M})
\end{displaymath}
In Scilab, il suo diagramma poli/zeri pu\`o essere ottenuto con i seguenti comandi:
\begin{verbatim}
// inizializziamo il sistema
clear;
// creiamo la funzione di trasferimento:
// definiamo il vettore di coefficienti del numeratore
numcoeff=[-1 0 0 0 0 0 0 0 1];
// creiamo il polinomio al numeratore
num=poly(numcoeff,'z','c');
// definiamo il vettore di coefficienti del denominatore
dencoeff=[0 0 0 0 0 0 0 0 8];
// creiamo il polinomio al denominatore
den=poly(dencoeff,'z','c');
// creiamo la funzione di trasferimento come sistema lineare
combsl=syslin('d',num,den);
// inizializziamo la finestra grafica
xbasc();
// visualizziamo il grafico poli/zeri
plzr(combsl);
\end{verbatim}
Otteniamo il risultato rappresentanto nella figura \ref{fir-comb8-pz}.
\input{images/fir-comb8-pz.tex}
\dessin{Grafico poli/zeri di un filtro COMB di ordine 8}{fir-comb8-pz}

La risposta in frequenza pu\`o essere determinata con i seguenti comandi:
\begin{verbatim}
// definiamo il range discreto delle frequenze
f=(-.5:.001:.5);
// calcoliamo la risposta in frequenza
hf=freq(num,den,exp(2*%pi*%i*f));
// inizializziamo la finestra grafica e 
// visualizziamo la risposta in frequenza
xbasc(); plot2d(f,abs(hf))
\end{verbatim}
Il grafico \`e mostrato in figura \ref{fir-comb8-fr}.
\input{images/fir-comb8-fr.tex} 
\dessin{Risposta in frequenza di un filtro COMB di ordine 8}{fir-comb8-fr}

\subsection*{Filtro passa-basso implementato con un filtro COMB}
\`E possibile modificare un filtro COMB in modo da filtrare determinate frequenze ed utilizzarlo quindi come un comune filtro passa-basso o passa-alto di facile implementazione ma scarsa efficacia.
Proviamo ad esempio ad eliminare il polo in $z=1$: la funzione di trasferimento del filtro ottenuto sar\`a
\begin{displaymath}
H(z)=\frac{1}{8} \frac{z^8 -1}{z^7 -z^6}
\end{displaymath}
Costruiamo il grafico poli/zeri di figura \ref{fir-comb8-lp-pz}:
\begin{verbatim}
clear;
numcoeff=[-1 0 0 0 0 0 0 0 1];
num=poly(numcoeff,'z','c');
dencoeff=[0 0 0 0 0 0 0 -1/8 1/8];
den=poly(dencoeff,'z','c');
combsl=syslin('d',num,den);
xbasc();plzr(combsl);
\end{verbatim}
\input{images/fir-comb8-lp-pz.tex}
\dessin{Grafico poli/zeri di un filtro passa-basso implementato con un filtro COMB di ordine 8}{fir-comb8-lp-pz}

La corrispondente risposta in frequenza si ottiene con i comandi 
\begin{verbatim}
f=(-.5:.001:.5);
hf=freq(num,den,exp(2*%pi*%i*f));
xbasc(); xset("font size",4); plot2d(f,abs(hf))
\end{verbatim}
ed \`e mostrata nella figura \ref{fir-comb8-lp-fr}.
\input{images/fir-comb8-lp-fr.tex}
\dessin{Risposta in frequenza di un filtro passa-basso implementato con un filtro COMB di ordine 8}{fir-comb8-lp-fr}

\section{Filtri realizzati con il metodo delle finestre}
Progettiamo un filtro con il metodo delle finestre: grazie a Scilab, \`e facile ottenerne i coefficienti usando la funzione \verb+wfir+, che richiede come parametri
\begin{itemize}
\item il tipo di filtro: \verb+lp+ (passa-basso), \verb+hp+ (passa-alto), \verb+bp+ (passa-banda), \verb+sb+ (taglia-banda)
\item l'ordine del filtro
\item una o due frequenze di taglio, in base al tipo di filtro scelto
\item il tipo di finestra da applicare: \verb+re+ (rettangolare), \verb+tr+ (triangolare), \verb+hm+ (Hamming), \verb+hn+ (Hanning), \verb+kr+ (Kaiser), \verb+ch+ (Chebyshev)
\item i parametri della finestra scelta  
\end{itemize}
e restituisce in output
\begin{itemize}
\item i coefficienti del filtro nel dominio del tempo
\item la risposta del filtro nel dominio della frequenza
\item i punti del dominio della frequenza nei quali la risposta viene valutata
\end{itemize}
Vediamone un paio di esempi.

\subsection*{Filtro passa-basso realizzato con una finestra rettangolare}
Creiamo un filtro passa-basso di ordine 20 con frequenza di taglio pari a 0.3 utilizzando una finestra rettangolare:
\begin{verbatim}
clear;
[valcoeff,filtamp,filtfreq]=wfir('lp',20,[.3 0],'re',[0 0]);
num=poly(valcoeff,'z','c');
xbasc();
// la funzione 'horner' valuta il polinomio num in 1/z
plzr(horner(num,1/%z));
\end{verbatim}
Il grafico poli/zeri risultante \`e mostrato in figura \ref{fir-wfir-lp-rect-pz}.
\input{images/fir-wfir-lp-rect-pz.tex}
\dessin{Grafico poli/zeri di un filtro passa-basso di ordine 20 realizzato con una finestra rettangolare}{fir-wfir-lp-rect-pz}

La risposta in frequenza (figura \ref{fir-wfir-lp-rect-fr}) pu\`o essere ottenuta con le seguenti istruzioni:
\begin{verbatim}
xbasc(); plot2d(filtfreq,filtamp);
\end{verbatim}
\input{images/fir-wfir-lp-rect-fr.tex}
\dessin{Risposta in frequenza di un filtro passa-basso di ordine 20 realizzato con una finestra rettangolare}{fir-wfir-lp-rect-fr}

\subsection*{Filtro passa-banda realizzato con una finestra di Hamming}
Creiamo un filtro passa-banda di ordine 20 con frequenze di taglio pari a 0.2 e 0.37 utilizzando una finestra di Hamming:
\begin{verbatim}
clear;
[coeff,amp,freq]=wfir('bp',20,[.2 .37],'hm',[0 0]);
num=poly(coeff,'z','c');
xbasc(); plzr(horner(num,1/%z));
\end{verbatim}
Il grafico poli/zeri del filtro corrispondente \`e mostrato in figura \ref{fir-wfir-bp-hm-pz}.
\input{images/fir-wfir-lp-rect-pz.tex}
\dessin{Grafico poli/zeri di un filtro passa-banda di ordine 20 realizzato con una finestra di Hamming}{fir-wfir-bp-hm-pz}

La risposta in frequenza (figura \ref{fir-wfir-bp-hm-fr}) pu\`o essere ottenuta con le seguenti istruzioni:
\begin{verbatim}
xbasc(); plot2d(freq,amp);
\end{verbatim}
\input{images/fir-wfir-bp-hm-fr.tex}
\dessin{Risposta in frequenza di un filtro passa-banda di ordine 20 realizzato con una finestra di Hamming}{fir-wfir-bp-hm-fr}

\section{Filtro passa-alto realizzato con il metodo minimax}
Progettiamo un filtro con il metodo dell'approssimazione minimax: grazie a Scilab, \`e facile ottenerne i coefficienti usando la funzione \verb+eqfir+, che richiede come parametri
\begin{itemize}
\item il numero dei punti di cui sar\`a composto il filtro risultante 
\item $M$ coppie di frequenze delimitanti le $M$ bande
\item $M$ magnitudini, un valore per ciascuna banda
\item $M$ valori rappresentanti il peso relativo dell'errore in ciascuna banda
\end{itemize}
e restituisce in output i coefficienti del filtro.\\
Vediamone un esempio: progettiamo un filtro di 33 elementi, con banda azzerata nell'intervallo (0, .22) e passante nell'intervallo (.25, .5).

\begin{verbatim}
clear;
// cerchiamo i coefficienti del filtro richiesto
hn=eqfir(33,[0 .22;.24 .5],[0 1],[.1 1]);
// creiamo ora la funzione di trasferimento 
filter=poly(hn,'z','c');
\end{verbatim}
Ottenuta la funzione di trasferimento, procediamo nel consueto modo per ottenere il grafico poli/zeri:
\begin{verbatim}
xbasc(); plzr(horner(filter,1/%z)); 
\end{verbatim}
Il risultato \`e visualizzato in figura \ref{fir-minimax-hp-pz}.
\input{images/fir-minimax-hp-pz.tex}
\dessin{Grafico poli/zeri di un filtro passa-alto realizzato con il metodo di approssimazione minimax}{fir-minimax-hp-pz}
La risposta in frequenza \`e calcolato con le seguenti righe di codice:
\begin{verbatim}
[hm,fr]=frmag(hn,256);
xbasc(); plot2d(fr,hm);
\end{verbatim}
Il suo grafico \`e in figura \ref{fir-minimax-hp-fr}.
\input{images/fir-minimax-hp-fr.tex}
\dessin{Risposta in frequenza di un filtro passa-alto realizzato con il metodo di approssimazione minimax}{fir-minimax-hp-fr}

