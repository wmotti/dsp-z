\chapter{Introduzione a Scicos}
SCICOS (SCIlab Connected Object Simulator) \`e una libreria di Scilab che permette di simulare sistemi dinamici, sia continui che discreti e avvalendosi di un editor grafico permette di costruire modelli tramite l'interconnessione di blocchi.\\
I blocchi rappresentano funzioni base o funzioni definibili dall'utente e agiscono come un sottosistema che viene attivato da un input e fornisce come output un segnale dipendente dall'ingresso e dal suo stato.\\
In Scicos ad ogni segnale \`e associato un insieme di indici di tempo, detti istanti di attivazione, in cui evolve il segnale (per istanti non di attivazione il segnale rimane costante).\\
I segnali vengono generati da blocchi pilotati da segnali di attivazione che portano il blocco ad adeguare la sua uscita in funzione dell'ingresso e del suo stato interno.

\section{L'editor di Scicos}
Per avviare l'editor grafico attivando la finestra principale di Scicos \`e necessario digitare \verb+scicos()+ dall'editor di Scilab; come si vedr\`a l'interfaccia grafica \`e intuitiva e permette di creare facilmente i diagrammi dei sistemi.\\
Per costruire un modello in Scicos si eseguiranno tipicamente le seguenti operazioni:
\begin{itemize}
\item aprire il menu \verb+edit+ e cliccare sulla voce \verb+palettes+: verranno mostrate diverse categorie di blocchi che rappresentano le componenti tipiche dei sistemi
\item copiare i blocchi nella finestra di Scicos, cliccando sul componente desiderato e posizionandolo sulla finestra principale con un altro click
\item connettere le porte di input e output dei blocchi, cliccando semplicemente sulle porte che si desiderano collegare. Nel caso risultasse necessario biforcare l'uscita di un blocco per connetterla a pi\`u componenti si deve creare un nuovo link, scegliendone l'origine sul link da biforcare
\item inserire nei diagrammi almeno un blocco per poter visualizzare o salvare gli esiti della simulazione, ad esempio \verb+Mscope+ o \verb+scrivi su file+
\end{itemize}

\section{Blocchi base}
I blocchi base disponibili in Scicos sono di tre tipi:
\begin{itemize}
\item \verb+Regular Basic Block+ \\
possono rappresentare variabli continue $x$ o discrete $z$;
\item \verb+Zero Crossing Basic Block+ \\
generano output solo se almeno un input regolare cambia segno.
L'evento generato pu\`o dipendere dalla combinazione degli eventi che hanno cambiato segno;
\item \verb+Synchro Basic Block+ \\
blocchi di sincronizzazione; possono avere pi\`u input o output di attivazione, ma hanno un solo input regolare; appena ricevono in input un segnale di attivazione generano un segnale di attivazione in output scegliendo in base all'input regolare.
\end{itemize}
Ogni blocco pu\`o avere due tipi di porte sia di input sia di output:
porte per input regolari, per input di attivazione, per output regolari e per output di attivazione.

\section{Impostazione dei parametri delle componenti}
I parametri dei blocchi possono essere modificati cos\`i da simulare fedelmente il sistema da realizzare. Per modificarli, \`e sufficiente cliccare due volte sul componente.

\section{Simulazione}
Una volta completato lo schema del sistema e dopo aver impostato i parametri di ogni blocco, selezionando l'opzione \verb+run+ dal men\`u \verb+simulate+ il programma comincia la simulazione in una nuova finestra.
La simulazione pu\`o essere interrotta premendo il pulsante \verb+stop+ nella finestra principale.

\section{Altre funzionalit\`a}
\begin{itemize}
\item salvare e caricare programmi Scicos
\item modificare l'aspetto dei blocchi
\item esportare e stampare i diagrammi e le simulazioni di Scicos
\item interagire con codice C e Fortran
\item interagire con file audio
\item visualizzare l'help riguardante i blocchi (selezionare \verb+help+ e poi cliccare sul componente desiderato)
\item modularizzare sistemi complessi in sub-sistemi, cos\`i da semplificare lo schema generale e favorire la riusabilit\`a delle componenti
\end{itemize}

\section{Esempio: simulazione di un filtro passa-basso}
Al fine di dimostrare un possibile impiego di Scicos, realizziamo una simulazione di un segnale composto da due sinusoidi filtrato attraverso un filtro passa-basso:
\begin{enumerate}
\item inseriamo i due generatori di sinusoidi e li colleghiamo ad un sommatore, ottenendo cos\`i il segnale di input
\item definiamo la funzione di trasferimento del filtro come rapporto tra due polinomi in $z$
\item aggiungiamo un clock per attivare il sistema ad intervalli stabiliti
\item poniamo in ingresso ad un oggetto \verb+MScope+ il segnale in ingresso prima e dopo il passaggio attraverso il filtro 
\end{enumerate}
%%%%%%%%%%%%%%%%%%%%%%%%%%%%
% Usage: -To include a Figure with a caption, insert the TWO following lines
%        in your Latex file:
% \input{This_file_name} 
% \dessin{The_caption}{The_label}
%         -To include just a picture, insert the lines 
%         between \fbox{\begin{picture}...  and \end{picture}} below 
%          
%%%%%%%%%%%%%%%%%%%%%%%%%%%%
 
 \long\def\Checksifdef#1#2#3{%
\expandafter\ifx\csname #1\endcsname\relax#2\else#3\fi}
\Checksifdef{Figdir}{\gdef\Figdir{}}{}
\def\dessin#1#2{
\begin{figure}[hbtp]
\begin{center}
%%%%%%%%%%%%%%%%%%%%%%%% 
%If you prefer cm, uncomment the following two lines
%\setlength{\unitlength}{1mm}
%\fbox{\begin{picture}(850.50,601.02)
%%%%%%%%%%%%%%%%%%%%%%%% 
\fbox{\begin{picture}(300.00,212.00)
%%%%%%%%%%%%%%%%%%%%%%%%%%%%%
% If you want to use epsfig, uncomment the following line 
% and comment the \special line 
%\epsfig{file=\Figdir a.eps,width=300.00pt,height=212.00pt}
%%%%%%%%%%%%%%%%%%%%%%%%%%%%%
\special{psfile=\Figdir Schema_filtro_passa-basso.eps hscale=100.00 vscale=100.00}
\end{picture}}
\end{center}
\caption{\label{#2}#1}
\end{figure}}

\dessin{Esempio di schema Scicos}{Schema_filtro_passa-basso}

Avviando la simulazione del sistema, si ottiene l'uscita in figura \ref{grafico-scicos}.

%%%%%%%%%%%%%%%%%%%%%%%%%%%%
% Usage: -To include a Figure with a caption, insert the TWO following lines
%        in your Latex file:
% \input{This_file_name} 
% \dessin{The_caption}{The_label}
%         -To include just a picture, insert the lines 
%         between \fbox{\begin{picture}...  and \end{picture}} below 
%          
%%%%%%%%%%%%%%%%%%%%%%%%%%%%
 
 \long\def\Checksifdef#1#2#3{%
\expandafter\ifx\csname #1\endcsname\relax#2\else#3\fi}
\Checksifdef{Figdir}{\gdef\Figdir{}}{}
\def\dessin#1#2{
\begin{figure}[hbtp]
\begin{center}
%If you prefer cm use the following two lines
%\setlength{\unitlength}{1mm}
%\fbox{\begin{picture}(850.50,601.02)
\fbox{\begin{picture}(300.00,212.00)
% if you want to use epsfig uncomment the following line 
% and comment the special line 
%\epsfig{file=\Figdir Grafico filtro passa-basso.eps,width=300.00pt,height=212.00pt}
\special{psfile=\Figdir grafico-scicos.eps hscale=100.00 vscale=100.00}
\end{picture}}
\end{center}
\caption{\label{#2}#1}
\end{figure}}

\dessin{Segnale in input e segnale filtrato}{grafico-scicos}
