%%%%% Preamble %%%%%
%\documentclass[a4paper,11pt]{article}
%\author{Walter Mottinelli}
%\title{Elaborazione numerica dei segnali con Scilab}
%\usepackage{latexsym}
%\usepackage{makeidx} % to generate indexes
%\usepackage[italian]{babel}
%\usepackage{ucs} % unicode encoding
%\usepackage[utf8]{inputenc}
%\usepackage{graphicx} % eps importing
%\pagestyle{headings}
%%%%%%%%%%%%%%%%%%%%
                                                                           
%\begin{document}
%\maketitle
%\tableofcontents
%\begin{abstract}
%\end{abstract}

\chapter{La trasformata $z$}
La trasformata $z$ \`e uno dei principali strumenti matematici per l'analisi e la sintesi di filtri digitali. Trova infatti applicazione in alcune importanti problematiche, ad esempio:
\begin{itemize}
\item \emph{Calcolo della funzione di trasferimento del sistema} \\
La funzione di trasferimento di un filtro \`e la trasformata $z$ della sua risposta all'impulso. \\
Il piano complesso $z$ \`e utilizzato per mostrare la configurazione dei poli e degli zeri della funzione di trasferimento: la loro disposizione, come vedremo, \`e indice di importanti propriet\`a del sistema.\\
Inoltre, i coefficienti del filtro digitale possono essere ricavati direttamente a partire dalla funzione di trasferimento.
\item \emph{Calcolo delle trasformate di Fourier} \\
La trasformata di Fourier corrisponde al calcolo della trasformata $z$ lungo la circonferenza di raggio unitario con centro nell'origine degli assi del piano complesso $z$. In particolare, la versione discreta delle trasformata di Fourier viene valutata su insiemi di punti equidistanti tra loro.
\end{itemize}
\section{Definizione matematica}
La trasformata $z$ di una sequenza $\{h[n]\}$ \`e definita dalla funzione nella variabile $z$ a valori complessi
\begin{displaymath}
H(z) = \sum_{n=-\infty}^\infty h(n)z^{-n}
\end{displaymath}
e dalla regione di convergenza, luogo dei punti in cui la sommatoria \`e finita. \\
Se la sequenza $\{h[n]\}$ in esame \`e la risposta all'impulso di un filtro digitale, $H(n)$ viene chiamata \emph{funzione di trasferimento del sistema}.
\subsection*{Esempio di calcolo della trasformata $z$}
Sia data la funzione
\begin{displaymath}
h(n)=r^n cos(\omega_0 n)
\end{displaymath}
per $n\ge 0$, la cui trasformata $z$ \`e uguale a 
\begin{displaymath}
H(z)=\sum_{n=0}^\infty r^n cos(\omega_0 n)z^{-n}
\end{displaymath}
Applicando la formula
\begin{displaymath}
cos \omega = \frac{e^{i\omega n} + e^{-i\omega n}}{2}
\end{displaymath}
derivata dall'identit\`a di Eulero $e^{i\omega}=cos\omega + isen\omega$ 
e la formula della somma infinita di elementi di una serie geometrica\\
$\sum_{n=0}^\infty a^n = \frac{1}{1-a}$ per $|a|<1$ \\
otteniamo:
\begin{eqnarray*}
H(z) & = & \sum_{n=0}^\infty r^{n} \frac{e^{i\omega_0 n}+e^{-i\omega_0 n}}{2}z^{-n}\\
     & = & \frac{1}{2} \left [ \sum_{n=0}^\infty (re^{i\omega_0 n}z^{-1})^n + \sum_{n=0}^\infty (re^{-i\omega_0 n}z^{-1})^n\right ] \\
     & = & \frac{1}{2} \left [ \frac{1}{1-re^{i \omega_0}z^{-1}} + \frac{1}{1-re^{-i \omega_0}z^{-1}} \right ]\\
     & = & \frac{1}{2} \left [ \frac{2-r(e^{i\omega_0}+e^{-i\omega_0}z^{-1})}{(1-re^{i \omega_0}z^{-1})(1-re^{-i \omega_0})z^{-1}} \right ] \\
     & = & \frac{1-rcos(\omega_0)z^{-1}}{1-2rcos(\omega_0)z^{-1}+r^2 z^{-2}}
\end{eqnarray*}

\section{Propriet\`a della trasformata $z$}
\begin{itemize}
\item linearit\`a\\ se $\{x(n)\}=\{ah_1(n)+bh_2(n)\}$, allora $X(z)=aH_1(z)+bH_2(z)$
\item traslazione\\ se $\{y(n)\}=\{x(n-n_0)\}$, allora $Y(z)=z^{-n_0}X(z)$
\item convoluzione\\ se $\{y(n)\}=\{h(n)\}\ast\{x(n)\}$, allora $Y(z)=H(z)X(z)$
\item scalatura (nel dominio $z$)\\ se $\{y(n)\}=\{a^n x(n)\}$, allora $Y(z)=X(a^{-1} z)$
\item derivazione (nel dominio $z$)\\ se $\{y(n)\}=\{nx(n)\}$, allora $Y(z)=z^{-1} \frac{dX}{dz^{-1}}$
\end{itemize}

\section{Trasformate $z$ comuni}
\begin{tabular}{l|l|l}

Segnale & Trasformata $z$ & ROC \\
\hline
$\delta [n]$ & $1$ & $\forall z$ \\
$\delta [n-k]$ & $z^{-k}$ & $\forall z$, con $z\ne 0$ se $k>0$ \\
$\delta [n-k]$ & $z^{-k}$ & $\forall z$, con $z\ne \infty$ se $k<0$ \\
$u[n]$ & $\frac{z}{z-1}$ & $|z| > 1$ \\ 
$-(u[-n-1])$ & $\frac{z}{z-1}$ & $|z| < 1$ \\
$nu[n]$ & $\frac{z}{(z-1)^2}$ & $|z| > 1$ \\
$n^2 u[n]$ & $\frac{z(z+1)}{(z-1)^3}$ & $|z| > 1$ \\
$n^3 u[n]$ & $\frac{z(z^2 +4z +1)}{(z-1)^4}$ & $|z| > 1$ \\
$\left ( -\left(\alpha^n\right ) \right)u[-n-1]$ & $\frac{z}{z-\alpha}$ & $|z|<|\alpha|$ \\
$\alpha^n u[n]$ & $\frac{z}{z-\alpha}$ & $|z|>|\alpha|$ \\
$n\alpha^n u[n]$ & $\frac{\alpha z}{(z-\alpha)^2}$ & $|z|>|\alpha|$ \\
$n^2\alpha^n u[n]$ & $\frac{\alpha z(z+\alpha)}{(z-\alpha)^3}$ & $|z|>|\alpha|$ \\
$\frac{\prod_{k=1}^m (n-k+1)}{\alpha^m m!}\alpha^n u[n]$ & $\frac{z}{(z-\alpha)^{m+1}}$ \\
$\gamma^n cos(\alpha n)u[n]$ & $\frac{z(z-\gamma cos(\alpha))}{z^2 - (2\gamma cos(\alpha))z+\gamma^2}$ & $|z|>|\alpha|$ \\
$\gamma^n sen(\alpha n)u[n]$ & $\frac{z\gamma sen(\alpha)}{z^2 - (2\gamma cos(\alpha))z+\gamma^2}$ & $|z|>|\alpha|$
\end{tabular}

\section{Poli e zeri}
La trasformata $H(z)$ \`e una funzione nella variabile a valori complessi $z$, che pu\`o essere rappresentanta su un piano con la parte reale sull'asse delle ascisse e la parte immaginaria sull'asse delle ordinate.
Alcuni punti del piano complesso hanno particolare importanza:
\begin{itemize}
\item gli \emph{zeri} sono i punti in cui $H(z)=0$ 
\item i \emph{poli} sono i punti in cui $H(z)=\infty$
\end{itemize}
Con un grafico tridimensionale \`e possibile evidenziare il comportamento della trasformata in questi punti: in figura \ref{pole-zero-3D} \`e rappresentato in funzione di $\Re(z)$ e $\Im(z)$ il modulo di $H(z)=z^{-1}-1$, che ha uno zero in $z=1$ (l'avvallamento) e un polo in $z=0$ (il picco).
%%%%%%%%%%%%%%%%%%%%%%%%%%%%
% Usage: -To include a Figure with a caption, insert the TWO following lines
%        in your Latex file:
% \input{This_file_name} 
% \dessin{The_caption}{The_label}
%         -To include just a picture, insert the lines 
%         between \fbox{\begin{picture}...  and \end{picture}} below 
%          
%%%%%%%%%%%%%%%%%%%%%%%%%%%%
 
 \long\def\Checksifdef#1#2#3{%
\expandafter\ifx\csname #1\endcsname\relax#2\else#3\fi}
\Checksifdef{Figdir}{\gdef\Figdir{}}{}
\def\dessin#1#2{
\begin{figure}[htbp]
\begin{center}
%%%%%%%%%%%%%%%%%%%%%%%% 
%If you prefer cm, uncomment the following two lines
%\setlength{\unitlength}{1mm}
%\fbox{\begin{picture}(850.50,601.02)
%%%%%%%%%%%%%%%%%%%%%%%% 
\fbox{\begin{picture}(300.00,212.00)
%%%%%%%%%%%%%%%%%%%%%%%%%%%%%
% If you want to use epsfig, uncomment the following line 
% and comment the \special line 
%\epsfig{file=\Figdir pole-zero-3D.eps,width=300.00pt,height=212.00pt}
%%%%%%%%%%%%%%%%%%%%%%%%%%%%%
\special{psfile=\Figdir pole-zero-3D.eps hscale=100.00 vscale=100.00}
\end{picture}}
\end{center}
\caption{\label{#2}#1}
\end{figure}}

\dessin{Modulo di $H(z)=z^{-1}-1$ in funzione di $\Re(z)$ e $\Im(z)$}{pole-zero-3D}

Poli e zeri possono essere indicati esplicitamente nell'equazione di $H(z)$: consideriamo ad esempio una funzione di trasferimento di un segnale digitale, normalmente rappresentata come rapporto di somme di prodotti
\begin{displaymath}
H(z)=\frac{\sum_{k=-N_F}^{N_P} {b_k}z^{-k}} {1-\sum_{k=1}^Ma_kz^{-k}}
\end{displaymath}
dove 
\begin{itemize}
\item $N_F$ \`e il numero dei campioni in input successivi a quello in esame;
\item $N_P$ \`e il numero dei campioni in input precedenti quello in esame;
\item $M$  \`e il numero dei campioni in output precedenti quello in esame.
\end{itemize}
A partire da questa equazione \`e possibile passare ad un rapporto di prodotti di termini 
\begin{displaymath}
H(z)=Az^{N_F} \frac{\prod_{k=1}^{N_P+N_F}(1-c_k z^{-1})}{\prod_{k=1}^M (1-d_k z^{-1})}
\end{displaymath}
dove 
\begin{itemize}
\item A \`e una costante reale;
\item $c_k$, per $1\leq k \leq N_P + N_F$, \`e il valore (eventualmente complesso) del $k-esimo$ zero;
\item $d_k$, per $1\leq k \leq M$, \`e il valore del $k-esimo$ polo.
\end{itemize}
Alcune considerazioni sulla seconda rappresentazione:
\begin{itemize}
\item ogni fattore del numeratore ($1-c_kz^{-1}$) genera uno zero in $z=c_k$ e un polo in $z=0$;
\item ogni fattore del denominatore genera un polo in $z=d_k$ e uno zero in $z=0$;
\item nel punto $z=0$, poli dei fattori del numeratore cancellano zeri dei fattori del denominatore;
\item considerando anche i punti di singolarit\`a in $z=\infty$, il numero dei poli \`e \emph{sempre} pari al numero degli zero.
\end{itemize}
In sintesi, la disposizione dei poli e degli zeri della trasformata $z$ data nella forma 
\begin{displaymath}
H(z)=Az^{N_F} \frac{\prod_{k=1}^{N_P+N_F}(1-c_k z^{-1})}{\prod_{k=1}^M (1-d_k z^{-1)}}
\end{displaymath}
\`e la seguente:
\begin{itemize}
\item zeri in:
\begin{itemize}
\item $z=c_k$ \qquad per $1\leq k \leq N_P + N_F$
\item $z=0\textrm{ }$ \qquad di molteplicit\`a $M$
\item $z=0\textrm{ }$ \qquad di molteplicit\`a $N_F$
\end{itemize}
\item poli in:
\begin{itemize}
\item $z=d_k$ \qquad per $1\leq k \leq M$
\item $z=0\textrm{ }$ \qquad di molteplicit\`a $N_P + N_F$
\item $z=\infty$ \qquad di molteplicit\`a $N_F$
\end{itemize}
\end{itemize}

\subsection*{Esempio 1}
\begin{eqnarray*}
H(z) & = & \frac{1-rcos\left (\omega_0\right )z^{-1}}{1-2rcos{\omega_0}z^{-1}+r^2 z^{-2}} \\
     & = & \frac{1-rcos\left (\omega_0\right )z^{-1}}{1-r\left (e^{i\omega_0}+e^{-i\omega_0}\right )z^{-1}+r^2 z^{-2}} \\
     & = & \frac{1-rcos\left (\omega_0\right )z^{-1}}{\left (1-re^{i\omega_0}z^{-1}\right )\left (1-re^{-i\omega_0}z^{-1}\right )}
\end{eqnarray*}
Il numeratore genera uno zero in $z=rcos(\omega_0)$ e un polo in $z=0$, mentre il denominatore genera due zeri in $z=0$ e un polo in $z=re^{\pm i \omega_0}$.

\subsection*{Esempio 2 - singolarit\`a di second'ordine}
\begin{eqnarray*}
H(z) & = & z^2 + 2z + 3 + 2z^{-1} + z^{-2} \\
     & = & \left (z + 1 + z^{-1} \right )^2 \\
     & = & z^2 \left (1-c_1 z^{-1}\right )^2\left (1-c_2 z^{-1}\right )^2
\end{eqnarray*}
dove
\begin{eqnarray*}
c_1 & = & \frac{-1 + i\sqrt{3}}{2} = e^{-i2\pi/3} \\
c_2 & = & \frac{-1 - i\sqrt{3}}{2} = e^{-i2\pi/3}
\end{eqnarray*}
Il primo fattore genera due poli in $z=\infty$ e due zeri in $z=0$.
I due fattori contenenti $z^{-1}$ generano ciascuno due zeri in $z=c_1$ e $z=c_2$ e due poli in $z=0$.
\subsection*{Esempio 3 - zeri disposti sulla circonferenza unitaria}
$H(z)=1-z^{-N}$ \'e un polinomio di ordine N, e ha radici 
\begin{eqnarray*}
z^N & = & re^{i(\theta+2m \pi)/N} \\
    & = & r^{1/N}e^{i(\theta + 2m\pi)/N}
\end{eqnarray*}
dove $m$ \`e un intero.
Vale
\begin{displaymath}
z^N=1=e^{i2m\pi}
\end{displaymath} 
quindi gli zeri sono in $z=e^{(i2m\pi)/N}$, per $m=0, 1, 2, \ldots, N-1$: la loro posizione \`e lungo la circonferenza di raggio unitario e centro nell'origine degli assi, a partire da $z=1$ e distanziati di un angolo $2\pi/N$.
\section{La regione di convergenza (ROC)}
La trasformata $z$ \`e definita solo per valori di $z$ in cui $\sum_{n=-\infty}^\infty |h(n)z^{-n}|<\infty$. \\
Poich\'e $H(z)$ va all'infinito nei punti in cui sono definiti i poli, questi non possono trovarsi nella ROC. \\
Per individuare la ROC \`e possibile procedere nel seguente modo:
\begin{itemize}
\item per i segnali di durata finita, la ROC \`e l'intero piano Z escluso 
\begin{itemize}
\item $z=0$ per segnali causali
\item $z=\infty$ per segnali non causali
\item $z=0$ e  $z=\infty$ per i segnali bilaterali
\end{itemize}
\item per i segnali di durata infinita, la ROC \`e
\begin{itemize}
\item nel caso di segnali causali, l'intero piano eccetto il cerchio di raggio $r_1$ pari alla distanza tra l'origine e il polo finito pi\`u lontano
\item nel caso di segnali non causali, il cerchio di raggio $r_2$ pari alla distanza tra l'origine e il polo finito pi\`u vicino
\item nel caso di segnali bilaterali, l'anello racchiuso dalle circonferenze di raggio $r_1$ e $r_2$
\end{itemize}
\end{itemize}

La ROC \`e indispensabile per ottenere il segnale nel dominio temporale a partire dalla sua trasformata $z$, in quanto non vi \`e corrispondenza biunivoca tra un segnale e la sola forma analitica della sua trasformata $z$: segnali diversi possono avere la stessa trasformata, e quindi gli stessi poli e zeri, ma regione di convergenza diversa, come mostrato nel prossimo paragrafo.

\subsection{Determinazione della ROC di un segnale causale}
Prendiamo in considerazione il segnale
\begin{displaymath}
x(n)=a^n u(n)
\end{displaymath}
con
\begin{displaymath}
u(n)= \left\{
\begin{array}{ll}
1 & \textrm{per } n\geq0\\
0 & \textrm{altrimenti}
\end{array}
\right.
\end{displaymath}
cio\`e l'analogo discreto del segnale gradino unitario.\\
Calcolando la trasformata $z$ del segnale ${x(n)}$ si ricava
\begin{displaymath}
X(z) = \sum_{n=-\infty}^\infty a^{n}u(n)z^{-n}=\frac{1}{1-az^{-1}}=\frac{z}{z-a}
\end{displaymath}
Analizzando la trasformata $z$ si rileva uno zero in $z=0$ e un polo in $z=a$: la regione di convergenza \`e cos\`i definita per $|z|>|a|$, essendo il segnale causale.
Il diagramma poli/zeri pu\`o essere ottenuto con Scilab tramite i seguenti comandi:
\begin{verbatim}
// zero della funzione di trasferimento del filtro
a=.5;
// creazione del numeratore
Ncoeff=[0,1];
N=poly(Ncoeff,'z','coeff');
// creazione del denominatore
Dcoeff=[-(a),1];
D=poly(Dcoeff,'z','coeff');
// creazione della funzione di trasferimento H(z)=N/D
Ztrasf=syslin('d',N,D);
// disegno del grafico dei poli/zeri
xbasc(); xset("font size",4); plzr(Ztrasf);
// disegno della circonferenza che delimita la ROC (r=1)
xarc(-0.5,0.5,1,1,0,360*64)
\end{verbatim}
Come risultato otteniamo il grafico rappresentato in figura \ref{ROC_01}, in cui la ROC \`e stata evidenziata con una colorazione pi\`u scura.\\\\
%%%%%%%%%%%%%%%%%%%%%%%%%%%%
% Usage: -To include a Figure with a caption, insert the TWO following lines
%        in your Latex file:
% \input{This_file_name} 
% \dessin{The_caption}{The_label}
%         -To include just a picture, insert the lines 
%         between \fbox{\begin{picture}...  and \end{picture}} below 
%          
%%%%%%%%%%%%%%%%%%%%%%%%%%%%
 
 \long\def\Checksifdef#1#2#3{%
\expandafter\ifx\csname #1\endcsname\relax#2\else#3\fi}
\Checksifdef{Figdir}{\gdef\Figdir{}}{}
\def\dessin#1#2{
\begin{figure}[hbtp]
\begin{center}
%%%%%%%%%%%%%%%%%%%%%%%% 
%If you prefer cm, uncomment the following two lines
%\setlength{\unitlength}{1mm}
%\fbox{\begin{picture}(850.50,601.02)
%%%%%%%%%%%%%%%%%%%%%%%% 
\begin{picture}(300.00,212.00)
%%%%%%%%%%%%%%%%%%%%%%%%%%%%%
% If you want to use epsfig, uncomment the following line 
% and comment the \special line 
%\epsfig{file=\Figdir a.eps,width=300.00pt,height=212.00pt}
%%%%%%%%%%%%%%%%%%%%%%%%%%%%%
\special{psfile=\Figdir ROC_01.eps hscale=100.00 vscale=100.00}
\end{picture}
\end{center}
\caption{\label{#2}#1}
\end{figure}}

\dessin{Grafico della ROC e dei poli/zeri di un segnale causale}{ROC_01}

\subsection{Determinazione della ROC di un segnale non causale}
Prendiamo in considerazione il segnale
\begin{displaymath}
x(n)=-a^nu(-n-1)
\end{displaymath}
Calcolando la trasformata $z$ del segnale ${x(n)}$ si ricava
\begin{eqnarray*}
X(z) & = & -\sum_{n=-\infty}^\infty a^{n}u(-n-1)z^{-n} \\
     & = & -\sum_{n=-\infty}^{-1} a^{n}z^{-n} \\
     & = & 1-\sum_{n=0}^\infty (a^{1}z)^{-n} \\
     & = & \frac{z}{z-a}
\end{eqnarray*}
Analizzando la trasformata $z$ si rileva uno zero in $z=0$ e un polo in $z=a$: la regione di convergenza \`e definita per $|z|<|a|$, essendo il segnale non causale. \\
Il diagramma poli/zeri pu\`o essere ottenuto con Scilab tramite i seguenti comandi:
\begin{verbatim}
// zero della funzione di trasferimento del filtro
a=.5;
// creazione del numeratore
Ncoeff=[0,1];
N=poly(Ncoeff,'z','coeff');
// creazione del denominatore
Dcoeff=[-(a),1];
D=poly(Dcoeff,'z','coeff');
// creazione della funzione di trasferimento H(z)=N/D
Ztrasf=syslin('d',N,D);
// disegno del grafico dei poli/zeri
xbasc(); xset("font size",4); plzr(Ztrasf);
// disegno della circonferenza che delimita la ROC (r=1)
xarc(-0.5,0.5,1,1,0,360*64)
\end{verbatim}
Come risultato otteniamo il grafico rappresentato in figura \ref{ROC_02}, in cui la ROC \`e stata evidenziata con una colorazione pi\`u scura.
%%%%%%%%%%%%%%%%%%%%%%%%%%%%
% Usage: -To include a Figure with a caption, insert the TWO following lines
%        in your Latex file:
% \input{This_file_name} 
% \dessin{The_caption}{The_label}
%         -To include just a picture, insert the lines 
%         between \fbox{\begin{picture}...  and \end{picture}} below 
%          
%%%%%%%%%%%%%%%%%%%%%%%%%%%%
 
 \long\def\Checksifdef#1#2#3{%
\expandafter\ifx\csname #1\endcsname\relax#2\else#3\fi}
\Checksifdef{Figdir}{\gdef\Figdir{}}{}
\def\dessin#1#2{
\begin{figure}[hbtp]
\begin{center}
%%%%%%%%%%%%%%%%%%%%%%%% 
%If you prefer cm, uncomment the following two lines
%\setlength{\unitlength}{1mm}
%\fbox{\begin{picture}(850.50,601.02)
%%%%%%%%%%%%%%%%%%%%%%%% 
\begin{picture}(300.00,212.00)
%%%%%%%%%%%%%%%%%%%%%%%%%%%%%
% If you want to use epsfig, uncomment the following line 
% and comment the \special line 
%\epsfig{file=\Figdir ROC_02.eps,width=300.00pt,height=212.00pt}
%%%%%%%%%%%%%%%%%%%%%%%%%%%%%
\special{psfile=\Figdir ROC_02.eps hscale=100.00 vscale=100.00}
\end{picture}
\end{center}
\caption{\label{#2}#1}
\end{figure}}

\dessin{Grafico della ROC e dei poli/zeri di un segnale non causale}{ROC_02}

\subsection{Determinazione della ROC di un segnale bilatero}
Prendiamo in considerazione il segnale
\begin{displaymath}
x(n)=\left(-\frac{1}{4}\right)^{n}u(n)-\left(\frac{1}{2}\right)^{n}u(-n-1)
\end{displaymath}
Calcoliamo la trasformata $z$ del segnale ${x(n)}$ scomponendolo in due parti:
\begin{eqnarray*}
\left(-\frac{1}{4}\right)^nu(n) & \longrightarrow & \frac{z}{z+\frac{1}{4}} \\
-\left(\frac{1}{2}\right)^n u(n) & \longrightarrow & \frac{z}{z-\frac{1}{2}}
\end{eqnarray*}
quindi la trasformata $z$ finale risulta essere
\begin{displaymath}
X(z) = \frac{z}{z+\frac{1}{4}}+\frac{z}{z-\frac{1}{2}}=\frac{z(2z-\frac{1}{8})}{(z+\frac{1}{4})(z-\frac{1}{2})}
\end{displaymath}
Analizzando la trasformata $z$ si rilevano due zeri $\left( z=0 \textrm{ e } z=\frac{1}{16} \right)$ e due poli $\left( z=-\frac{1}{4} \textrm{ e } z=\frac{1}{2}\right)$.\\
Utilizzando la propriet\`a di linearit\`a della trasformata $z$ possiamo concludere che la regione di convergenza \`e definita come l'intersezione delle regioni di convergenza delle due parti del segnale, quindi \`e uguale a $\frac{1}{4}<|z|<\frac{1}{2}$.\\
Il diagramma dei poli-zeri pu\`o essere ottenuto con Scilab tramite i seguenti comandi:
\begin{verbatim}
// creazione del numeratore
Ncoeff=[0,-(1/8),2];
N=poly(Ncoeff,'z','coeff');
// creazione del denominatore
Dcoeff=[-(1/8),-(1/4),1];
D=poly(Dcoeff,'z','coeff');
// creazione della trasformata Z
Ztrasf=syslin('d',N,D)
// stampa del grafico
xbasc(); xset("font size",4); plzr(Ztrasf);
// disegno della circonferenza che delimita la ROC
xarc(-.5,.5,1,1,0,360*64)	// Anello esterno
xarc(-.25,.25,.5,.5,0,360*64)	// Anello interno
\end{verbatim}
Come risultato otteniamo il grafico rappresentato in figura \ref{ROC_03}, in cui la ROC \`e stata evidenziata con una colorazione pi\`u scura.\\
Il diagramma del modulo e della fase del filtro pu\`o essere ottenuto con Scilab tramite i seguenti comandi:
\begin{verbatim}
// creazione del numeratore
Ncoeff=[0,-(1/8),2];
N=poly(Ncoeff,'z','coeff');
// creazione del denominatore
Dcoeff=[-(1/8),-(1/4),1];
D=poly(Dcoeff,'z','coeff');
// range di frequenze
f=(0:.01:.5);
// calcolo della risposta in frequenza
hf=freq(N,D,exp(2*%pi*%i*f));
xbasc(); xset("font size",4);
xsetech([0,0,1,.5]); plot2d(f,(abs(hf)))
xtitle("Risposta in frequenza del filtro")
xsetech([0,.5,1,.5]); plot2d(f,atan(imag(hf),real(hf)));
xtitle("Risposta in fase del filtro")
\end{verbatim}
Come risultato otteniamo il grafico rappresentato in figura \ref{Grafico_MagFase_3}.\\\\
%%%%%%%%%%%%%%%%%%%%%%%%%%%%
% Usage: -To include a Figure with a caption, insert the TWO following lines
%        in your Latex file:
% \input{This_file_name} 
% \dessin{The_caption}{The_label}
%         -To include just a picture, insert the lines 
%         between \fbox{\begin{picture}...  and \end{picture}} below 
%          
%%%%%%%%%%%%%%%%%%%%%%%%%%%%
 
 \long\def\Checksifdef#1#2#3{%
\expandafter\ifx\csname #1\endcsname\relax#2\else#3\fi}
\Checksifdef{Figdir}{\gdef\Figdir{}}{}
\def\dessin#1#2{
\begin{figure}[hbtp]
\begin{center}
%%%%%%%%%%%%%%%%%%%%%%%% 
%If you prefer cm, uncomment the following two lines
%\setlength{\unitlength}{1mm}
%\fbox{\begin{picture}(850.50,601.02)
%%%%%%%%%%%%%%%%%%%%%%%% 
\begin{picture}(300.00,212.00)
%%%%%%%%%%%%%%%%%%%%%%%%%%%%%
% If you want to use epsfig, uncomment the following line 
% and comment the \special line 
%\epsfig{file=\Figdir ROC_3.eps,width=300.00pt,height=212.00pt}
%%%%%%%%%%%%%%%%%%%%%%%%%%%%%
\special{psfile=\Figdir ROC_03.eps hscale=100.00 vscale=100.00}
\end{picture}
\end{center}
\caption{\label{#2}#1}
\end{figure}}

\dessin{Grafico della ROC e dei poli/zeri di un segnale bilatero}{ROC_03}
%%%%%%%%%%%%%%%%%%%%%%%%%%%%
% Usage: -To include a Figure with a caption, insert the TWO following lines
%        in your Latex file:
% \input{This_file_name} 
% \dessin{The_caption}{The_label}
%         -To include just a picture, insert the lines 
%         between \fbox{\begin{picture}...  and \end{picture}} below 
%          
%%%%%%%%%%%%%%%%%%%%%%%%%%%%
 
 \long\def\Checksifdef#1#2#3{%
\expandafter\ifx\csname #1\endcsname\relax#2\else#3\fi}
\Checksifdef{Figdir}{\gdef\Figdir{}}{}
\def\dessin#1#2{
\begin{figure}[hbtp]
\begin{center}
%%%%%%%%%%%%%%%%%%%%%%%% 
%If you prefer cm, uncomment the following two lines
%\setlength{\unitlength}{1mm}
%\fbox{\begin{picture}(850.50,601.02)
%%%%%%%%%%%%%%%%%%%%%%%% 
\fbox{\begin{picture}(300.00,212.00)
%%%%%%%%%%%%%%%%%%%%%%%%%%%%%
% If you want to use epsfig, uncomment the following line 
% and comment the \special line 
%\epsfig{file=\Figdir Grafico_MagFase_3.eps,width=300.00pt,height=212.00pt}
%%%%%%%%%%%%%%%%%%%%%%%%%%%%%
\special{psfile=\Figdir Grafico_MagFase_3.eps hscale=100.00 vscale=100.00}
\end{picture}}
\end{center}
\caption{\label{#2}#1}
\end{figure}}

\dessin{Risposta in frequenza e in fase di un filtro bilatero}{Grafico_MagFase_3}

\section{Calcolo della trasformata inversa}
La trasformata $z$ inversa permette di tornare dal piano $z$ al dominio d'origine: esistono quattro principali metodi per il calcolo dell'antitrasformata.
\subsection{Metodo delle funzioni parziali}
Uno dei metodi pi\`u semplici, e al tempo stesso pi\`u utili, per calcolare l'antitrasformata $z$ consiste nel decomporre la trasformata in frazioni parziali.
Consideriamo ad esempio la trasformata
\begin{displaymath}
X(z)=\frac{1}{1-\frac{3}{2}z^{-1}+\frac{1}{2}z^{-2}}
\end{displaymath}
Fattorizziamo il denominatore in modo da ottenere
\begin{displaymath}
X(z)=\frac{1}{(1-z^{-1})(1-\frac{1}{2}z^{-1})}
\end{displaymath}
$X(z)$ pu\`o quindi essere espressa come 
\begin{displaymath}
X(z)=\frac{A}{1-z^{-1}}+\frac{B}{1-\frac{1}{2}z^{-1}}
\end{displaymath}
con $A=2$ e $B=-1$.\\
Grazie alla propriet\`a di linearit\`a, le due frazioni possono essere invertite separatamente: sono due trasformate semplici (presenti nella tabella delle trasformata pi\`u comuni), quindi otteniamo
\begin{displaymath}
x(n)=Z^{-1}{X(z)}=2u(n)-\left ({\frac{1}{2}}\right )^n u(n)
\end{displaymath}
\subsection{Metodo della divisione}
A partire dalla funzione di trasferimento espressa come rapporto tra numeratore e denominatore, \`e possibile eseguire la divisione polinomiale ed ottenere quindi un unico polinomio: la sua trasformata inversa \`e una successione di impulsi pesati.
Riprendiamo l'esempio precedente: 
\begin{displaymath}
X(z)=\frac{1}{1-\frac{3}{2}z^{-1}+\frac{1}{2}z^{-2}}
\end{displaymath}
Effettuando la divisione si ottiene:
\begin{displaymath}
X(z)=1+ \frac{3}{2} z^{-1}+ \frac{7}{4} z^{-2}+ + \frac{15}{8} z^{-3} +\ldots
\end{displaymath}
la cui trasformata inversa \`e
\begin{displaymath}
x(n)=\delta(n)+ \frac{3}{2}\delta(n-1)+ \frac{7}{4}\delta(n-2)+ \frac{15}{8}\delta(n-3) + \ldots
\end{displaymath}

\subsection{Metodo del prodotto di Cauchy}
Analogamente al precedente, questo metodo consente di ottenere la trasformata inversa come somma di impulsi pesati e ritardati nel tempo.
Se $X(z)$ \`e nella forma
\begin{displaymath}
X(z)=\frac{b_0 + b_1 z^{-1}+\ldots + b_m z^{-m}}{a_0 + a_1 z^{-1}+\ldots + a_m z^{-m}}
\end{displaymath}
applicando il prodotto di Cauchy (il prodotto di convoluzione) e manipolando l'equazione in x(n) si ottiene l'equazione ricorsiva
\begin{displaymath}
x(n)=\frac{1}{a_0}\left[ b_n - \sum_{k=0}^{n-1} x(k) a_{n-k}\right]
\end{displaymath}
con $a_0 \neq 0$. \\
Nell'esempio precedente
\begin{displaymath}
X(z)=\frac{1}{1-\frac{3}{2}z^{-1}+\frac{1}{2}z^{-2}}
\end{displaymath}
i pesi sono
\begin{displaymath}
b_0 = 1, a_0=1, a_1=-\frac{3}{2}, a_2=\frac{1}{2}
\end{displaymath}
\`E possibile applicare l'equazione ricorsiva in x(n) e ottenere
\begin{displaymath}
x(0)=1, x(1)=\frac{3}{2}, x(2)=\frac{7}{4}, x(3)=\frac{15}{8}, \ldots
\end{displaymath}

\subsection{Metodo di integrazione}
Questo metodo \`e analiticamente il pi\`u complicato. \`E possibile dimostrare che esiste una relazione diretta tra l'$n-esimo$ elemento di una sequenza a tempo discreto ${h(n)}$ e la sua trasformata $z$, secondo la relazione:
\begin{displaymath}
x(n)=\frac{1}{2\pi i}\oint_C X(z)z^{n-1}dz
\end{displaymath}
dove l'integrazione \`e eseguita in senso antiorario lungo un percorso chiuso racchiudente l'origine degli assi ed interamente contenuto nella ROC.\\
Il calcolo diretto dell'integrale \`e spesso difficile, quindi viene utilizzato un secondo risultato, il \emph{teorema dei residui}:
\begin{displaymath}
h(n)=\sum_{i=1}^{N} \textrm{Res}[H(z)z^{n-1} \textrm{ nel polo }p_i]
\end{displaymath}
dove $\textrm{Res}[\cdot]$ \`e il \emph{residuo}.
Per definire il residuo, sia $H(z)z^{n-1}$ una funzione razionale in $z$ con un polo di ordine $m$ in $z=z_0$:
\begin{displaymath}
H(z)z^{n-1}=\frac{P(z)}{(z-z_0)^m}
\end{displaymath}
Il residuo di $H(z)z^{n-1}$ in $z=z_0$ \`e definito come
\begin{displaymath}
\textrm{Res}[H(z)z^{n-1} \textrm{ nel polo }p_i]=\left. \frac{1}{(m-1)!}\frac{d^{m-1}P(z)}{dz^{m-1}}\right |_{z=z_0}
\end{displaymath}
